% Options for packages loaded elsewhere
\PassOptionsToPackage{unicode}{hyperref}
\PassOptionsToPackage{hyphens}{url}
%
\documentclass[
]{article}
\usepackage{lmodern}
\usepackage{amssymb,amsmath}
\usepackage{ifxetex,ifluatex}
\ifnum 0\ifxetex 1\fi\ifluatex 1\fi=0 % if pdftex
  \usepackage[T1]{fontenc}
  \usepackage[utf8]{inputenc}
  \usepackage{textcomp} % provide euro and other symbols
\else % if luatex or xetex
  \usepackage{unicode-math}
  \defaultfontfeatures{Scale=MatchLowercase}
  \defaultfontfeatures[\rmfamily]{Ligatures=TeX,Scale=1}
\fi
% Use upquote if available, for straight quotes in verbatim environments
\IfFileExists{upquote.sty}{\usepackage{upquote}}{}
\IfFileExists{microtype.sty}{% use microtype if available
  \usepackage[]{microtype}
  \UseMicrotypeSet[protrusion]{basicmath} % disable protrusion for tt fonts
}{}
\makeatletter
\@ifundefined{KOMAClassName}{% if non-KOMA class
  \IfFileExists{parskip.sty}{%
    \usepackage{parskip}
  }{% else
    \setlength{\parindent}{0pt}
    \setlength{\parskip}{6pt plus 2pt minus 1pt}}
}{% if KOMA class
  \KOMAoptions{parskip=half}}
\makeatother
\usepackage{xcolor}
\IfFileExists{xurl.sty}{\usepackage{xurl}}{} % add URL line breaks if available
\IfFileExists{bookmark.sty}{\usepackage{bookmark}}{\usepackage{hyperref}}
\hypersetup{
  hidelinks,
  pdfcreator={LaTeX via pandoc}}
\urlstyle{same} % disable monospaced font for URLs
\usepackage[margin=1in]{geometry}
\usepackage{color}
\usepackage{fancyvrb}
\newcommand{\VerbBar}{|}
\newcommand{\VERB}{\Verb[commandchars=\\\{\}]}
\DefineVerbatimEnvironment{Highlighting}{Verbatim}{commandchars=\\\{\}}
% Add ',fontsize=\small' for more characters per line
\usepackage{framed}
\definecolor{shadecolor}{RGB}{248,248,248}
\newenvironment{Shaded}{\begin{snugshade}}{\end{snugshade}}
\newcommand{\AlertTok}[1]{\textcolor[rgb]{0.94,0.16,0.16}{#1}}
\newcommand{\AnnotationTok}[1]{\textcolor[rgb]{0.56,0.35,0.01}{\textbf{\textit{#1}}}}
\newcommand{\AttributeTok}[1]{\textcolor[rgb]{0.77,0.63,0.00}{#1}}
\newcommand{\BaseNTok}[1]{\textcolor[rgb]{0.00,0.00,0.81}{#1}}
\newcommand{\BuiltInTok}[1]{#1}
\newcommand{\CharTok}[1]{\textcolor[rgb]{0.31,0.60,0.02}{#1}}
\newcommand{\CommentTok}[1]{\textcolor[rgb]{0.56,0.35,0.01}{\textit{#1}}}
\newcommand{\CommentVarTok}[1]{\textcolor[rgb]{0.56,0.35,0.01}{\textbf{\textit{#1}}}}
\newcommand{\ConstantTok}[1]{\textcolor[rgb]{0.00,0.00,0.00}{#1}}
\newcommand{\ControlFlowTok}[1]{\textcolor[rgb]{0.13,0.29,0.53}{\textbf{#1}}}
\newcommand{\DataTypeTok}[1]{\textcolor[rgb]{0.13,0.29,0.53}{#1}}
\newcommand{\DecValTok}[1]{\textcolor[rgb]{0.00,0.00,0.81}{#1}}
\newcommand{\DocumentationTok}[1]{\textcolor[rgb]{0.56,0.35,0.01}{\textbf{\textit{#1}}}}
\newcommand{\ErrorTok}[1]{\textcolor[rgb]{0.64,0.00,0.00}{\textbf{#1}}}
\newcommand{\ExtensionTok}[1]{#1}
\newcommand{\FloatTok}[1]{\textcolor[rgb]{0.00,0.00,0.81}{#1}}
\newcommand{\FunctionTok}[1]{\textcolor[rgb]{0.00,0.00,0.00}{#1}}
\newcommand{\ImportTok}[1]{#1}
\newcommand{\InformationTok}[1]{\textcolor[rgb]{0.56,0.35,0.01}{\textbf{\textit{#1}}}}
\newcommand{\KeywordTok}[1]{\textcolor[rgb]{0.13,0.29,0.53}{\textbf{#1}}}
\newcommand{\NormalTok}[1]{#1}
\newcommand{\OperatorTok}[1]{\textcolor[rgb]{0.81,0.36,0.00}{\textbf{#1}}}
\newcommand{\OtherTok}[1]{\textcolor[rgb]{0.56,0.35,0.01}{#1}}
\newcommand{\PreprocessorTok}[1]{\textcolor[rgb]{0.56,0.35,0.01}{\textit{#1}}}
\newcommand{\RegionMarkerTok}[1]{#1}
\newcommand{\SpecialCharTok}[1]{\textcolor[rgb]{0.00,0.00,0.00}{#1}}
\newcommand{\SpecialStringTok}[1]{\textcolor[rgb]{0.31,0.60,0.02}{#1}}
\newcommand{\StringTok}[1]{\textcolor[rgb]{0.31,0.60,0.02}{#1}}
\newcommand{\VariableTok}[1]{\textcolor[rgb]{0.00,0.00,0.00}{#1}}
\newcommand{\VerbatimStringTok}[1]{\textcolor[rgb]{0.31,0.60,0.02}{#1}}
\newcommand{\WarningTok}[1]{\textcolor[rgb]{0.56,0.35,0.01}{\textbf{\textit{#1}}}}
\usepackage{graphicx}
\makeatletter
\def\maxwidth{\ifdim\Gin@nat@width>\linewidth\linewidth\else\Gin@nat@width\fi}
\def\maxheight{\ifdim\Gin@nat@height>\textheight\textheight\else\Gin@nat@height\fi}
\makeatother
% Scale images if necessary, so that they will not overflow the page
% margins by default, and it is still possible to overwrite the defaults
% using explicit options in \includegraphics[width, height, ...]{}
\setkeys{Gin}{width=\maxwidth,height=\maxheight,keepaspectratio}
% Set default figure placement to htbp
\makeatletter
\def\fps@figure{htbp}
\makeatother
\setlength{\emergencystretch}{3em} % prevent overfull lines
\providecommand{\tightlist}{%
  \setlength{\itemsep}{0pt}\setlength{\parskip}{0pt}}
\setcounter{secnumdepth}{-\maxdimen} % remove section numbering

\author{}
\date{\vspace{-2.5em}}

\begin{document}

\hypertarget{teoria-wspuxf3ux142bieux17cnoux15bci-lab4---sprawozdanie}{%
\section{Teoria Współbieżności Lab4 -
sprawozdanie}\label{teoria-wspuxf3ux142bieux17cnoux15bci-lab4---sprawozdanie}}

\hypertarget{autor-michaux142-flak}{%
\subsection{Autor: Michał Flak}\label{autor-michaux142-flak}}

\hypertarget{ux107wiczenie}{%
\subsubsection{Ćwiczenie}\label{ux107wiczenie}}

Badanie efektywności mechanizmów synchronizacji.

Problem czytelników i pisarzy proszę rozwiązać przy pomocy:

\begin{itemize}
\tightlist
\item
  Semaforów
\item
  Zmiennych warunkowych
\end{itemize}

Rozważyć przypadki z faworyzowaniem czytelników, pisarzy oraz z użyciem
kolejki FIFO.

Proszę wykonać pomiary dla każdego rozwiązania dla różnej ilości
czytelników (10-100) i pisarzy (od 1 do 10).

W sprawozdaniu proszę narysować wykres czasu w zależności od ilości
wątków i go zinterpretować.

\hypertarget{implementacja}{%
\subsection{Implementacja}\label{implementacja}}

Napisano klasę abstrakcyjną, po której będą dziedziczyć poszczególne
rozwiązania. Przyjmuje ona w konstruktorze ilość powtórzeń pętli
czytaj/pisz, jak również długość \texttt{sleep} symulującego działanie.
Klasa zawiera metodę \texttt{start} która przyjmuje w argumentach
\texttt{int\ readers,\ int\ writers} - czyli ilość tworzonych wątków.
Watki uruchamiane są w pętli jeden po drugim, po wszystkim metoda czeka
na zakończenie wszystkich wątków.

\begin{Shaded}
\begin{Highlighting}[]
\KeywordTok{abstract} \KeywordTok{class}\NormalTok{ AbstractReadersWriters}
\NormalTok{\{}
    \KeywordTok{protected} \DataTypeTok{int}\NormalTok{ repeat;}
    \KeywordTok{protected} \DataTypeTok{int}\NormalTok{ sleep;}

    \KeywordTok{public} \FunctionTok{AbstractReadersWriters}\NormalTok{(}\DataTypeTok{int}\NormalTok{ repeat, }\DataTypeTok{int}\NormalTok{ sleep) \{}
        \KeywordTok{this}\NormalTok{.}\FunctionTok{repeat}\NormalTok{ = repeat;}
        \KeywordTok{this}\NormalTok{.}\FunctionTok{sleep}\NormalTok{ = sleep;}
\NormalTok{    \}}


    \DataTypeTok{void} \FunctionTok{start}\NormalTok{(}\DataTypeTok{int}\NormalTok{ readers, }\DataTypeTok{int}\NormalTok{ writers)}
\NormalTok{    \{}
        \BuiltInTok{List}\NormalTok{\textless{}}\BuiltInTok{Thread}\NormalTok{\textgreater{} threads = }\KeywordTok{new} \BuiltInTok{ArrayList}\NormalTok{\textless{}\textgreater{}();}
        \KeywordTok{for}\NormalTok{ (}\DataTypeTok{int}\NormalTok{ i = }\DecValTok{1}\NormalTok{; i \textless{}= readers; i++) \{}
            \BuiltInTok{Thread}\NormalTok{ t = }\KeywordTok{new} \BuiltInTok{Thread}\NormalTok{(}\FunctionTok{createReader}\NormalTok{());}
\NormalTok{            t.}\FunctionTok{setName}\NormalTok{(}\BuiltInTok{String}\FunctionTok{.format}\NormalTok{(}\StringTok{"\#}\SpecialCharTok{\%03d}\StringTok{"}\NormalTok{, i));}
\NormalTok{            threads.}\FunctionTok{add}\NormalTok{(t);}
\NormalTok{        \}}
        \KeywordTok{for}\NormalTok{ (}\DataTypeTok{int}\NormalTok{ i = }\DecValTok{1}\NormalTok{; i \textless{}= writers; i++) \{}
            \BuiltInTok{Thread}\NormalTok{ t = }\KeywordTok{new} \BuiltInTok{Thread}\NormalTok{(}\FunctionTok{createWriter}\NormalTok{());}
\NormalTok{            t.}\FunctionTok{setName}\NormalTok{(}\BuiltInTok{String}\FunctionTok{.format}\NormalTok{(}\StringTok{"\#}\SpecialCharTok{\%03d}\StringTok{"}\NormalTok{, i));}
\NormalTok{            threads.}\FunctionTok{add}\NormalTok{(t);}
\NormalTok{        \}}

        \KeywordTok{for}\NormalTok{ (}\BuiltInTok{Thread}\NormalTok{ t : threads) \{}
\NormalTok{            t.}\FunctionTok{start}\NormalTok{();}
\NormalTok{        \}}
        \KeywordTok{for}\NormalTok{ (}\BuiltInTok{Thread}\NormalTok{ t : threads) \{}
            \KeywordTok{try}\NormalTok{ \{}
\NormalTok{                t.}\FunctionTok{join}\NormalTok{(); }\CommentTok{// wait for all to finish}
\NormalTok{            \} }\KeywordTok{catch}\NormalTok{ (}\BuiltInTok{InterruptedException}\NormalTok{ e) \{}
\NormalTok{                e.}\FunctionTok{printStackTrace}\NormalTok{();}
\NormalTok{            \}}
\NormalTok{        \}}
\NormalTok{    \}}

    \KeywordTok{abstract} \BuiltInTok{Reader} \FunctionTok{createReader}\NormalTok{();}

    \KeywordTok{abstract} \BuiltInTok{Writer} \FunctionTok{createWriter}\NormalTok{();}

    \KeywordTok{abstract} \KeywordTok{class} \BuiltInTok{Reader} \KeywordTok{implements} \BuiltInTok{Runnable}
\NormalTok{    \{}
        \AttributeTok{@Override}
        \KeywordTok{public} \DataTypeTok{void} \FunctionTok{run}\NormalTok{()}
\NormalTok{        \{}
            \KeywordTok{for}\NormalTok{ (}\DataTypeTok{int}\NormalTok{ i = }\DecValTok{0}\NormalTok{; i \textless{} repeat; i++) \{}
                \KeywordTok{try}\NormalTok{ \{}
                    \FunctionTok{before}\NormalTok{();}
                    \FunctionTok{read}\NormalTok{();}
                    \FunctionTok{after}\NormalTok{();}
\NormalTok{                \} }\KeywordTok{catch}\NormalTok{ (}\BuiltInTok{InterruptedException}\NormalTok{ e) \{}
                    \BuiltInTok{System}\NormalTok{.}\FunctionTok{err}\NormalTok{.}\FunctionTok{println}\NormalTok{(e.}\FunctionTok{getMessage}\NormalTok{());}
\NormalTok{                \}}
\NormalTok{            \}}
\NormalTok{        \}}

        \DataTypeTok{void} \FunctionTok{read}\NormalTok{() }\KeywordTok{throws} \BuiltInTok{InterruptedException}
\NormalTok{        \{}
            \BuiltInTok{Thread}\NormalTok{.}\FunctionTok{sleep}\NormalTok{(sleep);}
\NormalTok{        \}}

        \KeywordTok{abstract} \DataTypeTok{void} \FunctionTok{before}\NormalTok{() }\KeywordTok{throws} \BuiltInTok{InterruptedException}\NormalTok{;}

        \KeywordTok{abstract} \DataTypeTok{void} \FunctionTok{after}\NormalTok{() }\KeywordTok{throws} \BuiltInTok{InterruptedException}\NormalTok{;}
\NormalTok{    \}}

    \KeywordTok{abstract} \KeywordTok{class} \BuiltInTok{Writer} \KeywordTok{implements} \BuiltInTok{Runnable}
\NormalTok{    \{}
        \AttributeTok{@Override}
        \KeywordTok{public} \DataTypeTok{void} \FunctionTok{run}\NormalTok{()}
\NormalTok{        \{}
            \KeywordTok{for}\NormalTok{ (}\DataTypeTok{int}\NormalTok{ i = }\DecValTok{0}\NormalTok{; i \textless{} repeat; i++) \{}
                \KeywordTok{try}\NormalTok{ \{}
                    \FunctionTok{before}\NormalTok{();}
                    \FunctionTok{write}\NormalTok{();}
                    \FunctionTok{after}\NormalTok{();}
\NormalTok{                \} }\KeywordTok{catch}\NormalTok{ (}\BuiltInTok{InterruptedException}\NormalTok{ e) \{}
                    \BuiltInTok{System}\NormalTok{.}\FunctionTok{err}\NormalTok{.}\FunctionTok{println}\NormalTok{(e.}\FunctionTok{getMessage}\NormalTok{());}
\NormalTok{                \}}
\NormalTok{            \}}
\NormalTok{        \}}

        \DataTypeTok{void} \FunctionTok{write}\NormalTok{() }\KeywordTok{throws} \BuiltInTok{InterruptedException}
\NormalTok{        \{}
            \BuiltInTok{Thread}\NormalTok{.}\FunctionTok{sleep}\NormalTok{(sleep);}
\NormalTok{        \}}

        \KeywordTok{abstract} \DataTypeTok{void} \FunctionTok{before}\NormalTok{() }\KeywordTok{throws} \BuiltInTok{InterruptedException}\NormalTok{;}

        \KeywordTok{abstract} \DataTypeTok{void} \FunctionTok{after}\NormalTok{() }\KeywordTok{throws} \BuiltInTok{InterruptedException}\NormalTok{;}
\NormalTok{    \}}
\NormalTok{\}}
\end{Highlighting}
\end{Shaded}

\hypertarget{main}{%
\subsubsection{Main}\label{main}}

Metoda Main tworzy pliki o nazwie \texttt{n\_writers.txt}, który ma
następującą zawartość:

\begin{verbatim}
nReaders RWSRP RWSWP RWSQ RWC
10       396   399   444  397
20       393   403   516  409
30       400   412   510  409
...
100      398   441   621  409
\end{verbatim}

gdzie nReaders - liczba wątków czytelników, pozostałe kolumny
przedstawiają zaś czas w milisekundach rozwiązania problemu dla
odpowiednich metod.

\begin{Shaded}
\begin{Highlighting}[]
\KeywordTok{public} \KeywordTok{class}\NormalTok{ Main \{}
    \KeywordTok{public} \DataTypeTok{static} \DataTypeTok{void} \FunctionTok{main}\NormalTok{(}\BuiltInTok{String}\NormalTok{[] args) \{}

        \DataTypeTok{int}\NormalTok{ repeat = }\DecValTok{100}\NormalTok{;}
        \DataTypeTok{int}\NormalTok{ sleep = }\DecValTok{1}\NormalTok{;}

        \BuiltInTok{Map}\NormalTok{\textless{}}\BuiltInTok{String}\NormalTok{, AbstractReadersWriters\textgreater{} solutions = }\KeywordTok{new} \BuiltInTok{HashMap}\NormalTok{\textless{}\textgreater{}();}
\NormalTok{        solutions.}\FunctionTok{put}\NormalTok{(}\StringTok{"RWSRP"}\NormalTok{, }\KeywordTok{new} \FunctionTok{ReadersWritersSemReadersPreference}\NormalTok{(repeat, sleep));}
\NormalTok{        solutions.}\FunctionTok{put}\NormalTok{(}\StringTok{"RWSWP"}\NormalTok{, }\KeywordTok{new} \FunctionTok{ReadersWritersSemWritersPreference}\NormalTok{(repeat, sleep));}
\NormalTok{        solutions.}\FunctionTok{put}\NormalTok{(}\StringTok{"RWSQ"}\NormalTok{, }\KeywordTok{new} \FunctionTok{ReadersWritersSemQueue}\NormalTok{(repeat, sleep));}
\NormalTok{        solutions.}\FunctionTok{put}\NormalTok{(}\StringTok{"RWC"}\NormalTok{, }\KeywordTok{new} \FunctionTok{ReadersWritersCond}\NormalTok{(repeat, sleep));}
        \KeywordTok{try}\NormalTok{ \{}
            \KeywordTok{for}\NormalTok{(}\DataTypeTok{int}\NormalTok{ nWriters = }\DecValTok{1}\NormalTok{; nWriters \textless{}= }\DecValTok{10}\NormalTok{; nWriters++)\{}
                \BuiltInTok{FileWriter}\NormalTok{ myWriter = }\KeywordTok{null}\NormalTok{;}
\NormalTok{                myWriter = }\KeywordTok{new} \BuiltInTok{FileWriter}\NormalTok{(}\BuiltInTok{String}\FunctionTok{.format}\NormalTok{(}\StringTok{"}\SpecialCharTok{\%d}\StringTok{\_writers.txt"}\NormalTok{, nWriters));}
\NormalTok{                myWriter.}\FunctionTok{write}\NormalTok{(}\BuiltInTok{String}\FunctionTok{.format}\NormalTok{(}
                        \StringTok{"}\SpecialCharTok{\%s}\StringTok{ }\SpecialCharTok{\%s}\StringTok{ }\SpecialCharTok{\%s}\StringTok{ }\SpecialCharTok{\%s}\StringTok{ }\SpecialCharTok{\%s\textbackslash{}n}\StringTok{"}\NormalTok{,}
                        \StringTok{"nReaders"}\NormalTok{,}
                        \StringTok{"RWSRP"}\NormalTok{,}
                        \StringTok{"RWSWP"}\NormalTok{,}
                        \StringTok{"RWSQ"}\NormalTok{,}
                        \StringTok{"RWC"}
\NormalTok{                ));}

                \KeywordTok{for}\NormalTok{(}\DataTypeTok{int}\NormalTok{ nReaders = }\DecValTok{10}\NormalTok{; nReaders \textless{}= }\DecValTok{100}\NormalTok{; nReaders+=}\DecValTok{10}\NormalTok{)\{}
                    \BuiltInTok{Map}\NormalTok{\textless{}}\BuiltInTok{String}\NormalTok{, }\BuiltInTok{Long}\NormalTok{\textgreater{} results = }\KeywordTok{new} \BuiltInTok{HashMap}\NormalTok{\textless{}\textgreater{}();}

                    \DataTypeTok{int}\NormalTok{ finalNReaders = nReaders;}
                    \DataTypeTok{int}\NormalTok{ finalNWriters = nWriters;}
\NormalTok{                    solutions.}\FunctionTok{forEach}\NormalTok{((name, solution) {-}\textgreater{} \{}

                        \DataTypeTok{long}\NormalTok{ start = }\BuiltInTok{System}\NormalTok{.}\FunctionTok{currentTimeMillis}\NormalTok{();}
\NormalTok{                        solution.}\FunctionTok{start}\NormalTok{(finalNReaders, finalNWriters);}
                        \DataTypeTok{long}\NormalTok{ stop = }\BuiltInTok{System}\NormalTok{.}\FunctionTok{currentTimeMillis}\NormalTok{();}
\NormalTok{                        results.}\FunctionTok{put}\NormalTok{(name, stop {-} start);}
                        \BuiltInTok{System}\NormalTok{.}\FunctionTok{out.printf}\NormalTok{(}\StringTok{"}\SpecialCharTok{\%s\textbackslash{}t\%d\textbackslash{}t\%d\textbackslash{}t\%d\textbackslash{}n}\StringTok{"}\NormalTok{, name, finalNWriters, finalNReaders, stop {-} start);}
\NormalTok{                    \});}
\NormalTok{                    myWriter.}\FunctionTok{write}\NormalTok{(}\BuiltInTok{String}\FunctionTok{.format}\NormalTok{(}
                            \StringTok{"}\SpecialCharTok{\%d}\StringTok{ }\SpecialCharTok{\%d}\StringTok{ }\SpecialCharTok{\%d}\StringTok{ }\SpecialCharTok{\%d}\StringTok{ }\SpecialCharTok{\%d\textbackslash{}n}\StringTok{"}\NormalTok{,}
\NormalTok{                            nReaders,}
\NormalTok{                            results.}\FunctionTok{get}\NormalTok{(}\StringTok{"RWSRP"}\NormalTok{),}
\NormalTok{                            results.}\FunctionTok{get}\NormalTok{(}\StringTok{"RWSWP"}\NormalTok{),}
\NormalTok{                            results.}\FunctionTok{get}\NormalTok{(}\StringTok{"RWSQ"}\NormalTok{),}
\NormalTok{                            results.}\FunctionTok{get}\NormalTok{(}\StringTok{"RWC"}\NormalTok{)}
\NormalTok{                    ));}
\NormalTok{                \}}
\NormalTok{                myWriter.}\FunctionTok{close}\NormalTok{();}
\NormalTok{            \}}
\NormalTok{        \} }\KeywordTok{catch}\NormalTok{ (}\BuiltInTok{IOException}\NormalTok{ e) \{}
\NormalTok{            e.}\FunctionTok{printStackTrace}\NormalTok{();}
\NormalTok{        \}}
\NormalTok{    \}}
\NormalTok{\}}
\end{Highlighting}
\end{Shaded}

\hypertarget{rwsrp---readerswriterssemreaderspreference}{%
\subsubsection{1. RWSRP -
ReadersWritersSemReadersPreference}\label{rwsrp---readerswriterssemreaderspreference}}

\begin{Shaded}
\begin{Highlighting}[]
\KeywordTok{class}\NormalTok{ ReadersWritersSemReadersPreference }\KeywordTok{extends}\NormalTok{ AbstractReadersWriters}
\NormalTok{\{}
    \KeywordTok{private} \DataTypeTok{int}\NormalTok{ readCount = }\DecValTok{0}\NormalTok{;}
    \KeywordTok{private} \BuiltInTok{Semaphore}\NormalTok{ resource = }\KeywordTok{new} \BuiltInTok{Semaphore}\NormalTok{(}\DecValTok{1}\NormalTok{);}
    \KeywordTok{private} \BuiltInTok{Semaphore}\NormalTok{ rmutex = }\KeywordTok{new} \BuiltInTok{Semaphore}\NormalTok{(}\DecValTok{1}\NormalTok{);}

    \KeywordTok{public} \FunctionTok{ReadersWritersSemReadersPreference}\NormalTok{(}\DataTypeTok{int}\NormalTok{ repeat, }\DataTypeTok{int}\NormalTok{ sleep) \{}
        \KeywordTok{super}\NormalTok{(repeat, sleep);}
\NormalTok{    \}}

    \BuiltInTok{Reader} \FunctionTok{createReader}\NormalTok{()\{}
        \KeywordTok{return} \KeywordTok{new} \FunctionTok{ReaderSem}\NormalTok{();}
\NormalTok{    \}}

    \BuiltInTok{Writer} \FunctionTok{createWriter}\NormalTok{()\{}
        \KeywordTok{return} \KeywordTok{new} \FunctionTok{WriterSem}\NormalTok{();}
\NormalTok{    \}}

    \KeywordTok{class}\NormalTok{ ReaderSem }\KeywordTok{extends} \BuiltInTok{Reader}\NormalTok{ \{}

        \AttributeTok{@Override}
        \DataTypeTok{void} \FunctionTok{before}\NormalTok{() }\KeywordTok{throws} \BuiltInTok{InterruptedException}\NormalTok{ \{}
\NormalTok{            rmutex.}\FunctionTok{acquire}\NormalTok{();}
\NormalTok{            readCount++;}
            \KeywordTok{if}\NormalTok{(readCount == }\DecValTok{1}\NormalTok{) \{}
\NormalTok{                resource.}\FunctionTok{acquire}\NormalTok{();}
\NormalTok{            \}}
\NormalTok{            rmutex.}\FunctionTok{release}\NormalTok{();}
\NormalTok{        \}}

        \AttributeTok{@Override}
        \DataTypeTok{void} \FunctionTok{after}\NormalTok{() }\KeywordTok{throws} \BuiltInTok{InterruptedException}\NormalTok{ \{}
\NormalTok{            rmutex.}\FunctionTok{acquire}\NormalTok{();}
\NormalTok{            readCount{-}{-};}
            \KeywordTok{if}\NormalTok{(readCount == }\DecValTok{0}\NormalTok{)\{}
\NormalTok{                resource.}\FunctionTok{release}\NormalTok{();}
\NormalTok{            \}}
\NormalTok{            rmutex.}\FunctionTok{release}\NormalTok{();}
\NormalTok{        \}}
\NormalTok{    \}}

    \KeywordTok{class}\NormalTok{ WriterSem }\KeywordTok{extends} \BuiltInTok{Writer}\NormalTok{ \{}

        \AttributeTok{@Override}
        \DataTypeTok{void} \FunctionTok{before}\NormalTok{() }\KeywordTok{throws} \BuiltInTok{InterruptedException}\NormalTok{ \{}
\NormalTok{            resource.}\FunctionTok{acquire}\NormalTok{();}
\NormalTok{        \}}

        \AttributeTok{@Override}
        \DataTypeTok{void} \FunctionTok{after}\NormalTok{() }\KeywordTok{throws} \BuiltInTok{InterruptedException}\NormalTok{ \{}
\NormalTok{            resource.}\FunctionTok{release}\NormalTok{();}
\NormalTok{        \}}
\NormalTok{    \}}
\NormalTok{\}}
\end{Highlighting}
\end{Shaded}

\hypertarget{rwswp---readerswriterssemwriterspreference}{%
\subsubsection{2. RWSWP -
ReadersWritersSemWritersPreference}\label{rwswp---readerswriterssemwriterspreference}}

\begin{Shaded}
\begin{Highlighting}[]
\KeywordTok{class}\NormalTok{ ReadersWritersSemWritersPreference }\KeywordTok{extends}\NormalTok{ AbstractReadersWriters}
\NormalTok{\{}
    \KeywordTok{private} \DataTypeTok{int}\NormalTok{ readCount = }\DecValTok{0}\NormalTok{;}
    \KeywordTok{private} \DataTypeTok{int}\NormalTok{ writeCount = }\DecValTok{0}\NormalTok{;}
    \KeywordTok{private} \BuiltInTok{Semaphore}\NormalTok{ resource = }\KeywordTok{new} \BuiltInTok{Semaphore}\NormalTok{(}\DecValTok{1}\NormalTok{);}
    \KeywordTok{private} \BuiltInTok{Semaphore}\NormalTok{ rmutex = }\KeywordTok{new} \BuiltInTok{Semaphore}\NormalTok{(}\DecValTok{1}\NormalTok{);}
    \KeywordTok{private} \BuiltInTok{Semaphore}\NormalTok{ wmutex = }\KeywordTok{new} \BuiltInTok{Semaphore}\NormalTok{(}\DecValTok{1}\NormalTok{);}
    \KeywordTok{private} \BuiltInTok{Semaphore}\NormalTok{ readTry = }\KeywordTok{new} \BuiltInTok{Semaphore}\NormalTok{(}\DecValTok{1}\NormalTok{);}

    \KeywordTok{public} \FunctionTok{ReadersWritersSemWritersPreference}\NormalTok{(}\DataTypeTok{int}\NormalTok{ repeat, }\DataTypeTok{int}\NormalTok{ sleep) \{}
        \KeywordTok{super}\NormalTok{(repeat, sleep);}
\NormalTok{    \}}


    \BuiltInTok{Reader} \FunctionTok{createReader}\NormalTok{()\{}
        \KeywordTok{return} \KeywordTok{new} \FunctionTok{ReaderSem}\NormalTok{();}
\NormalTok{    \}}

    \BuiltInTok{Writer} \FunctionTok{createWriter}\NormalTok{()\{}
        \KeywordTok{return} \KeywordTok{new} \FunctionTok{WriterSem}\NormalTok{();}
\NormalTok{    \}}

    \KeywordTok{class}\NormalTok{ ReaderSem }\KeywordTok{extends} \BuiltInTok{Reader}\NormalTok{ \{}

        \AttributeTok{@Override}
        \DataTypeTok{void} \FunctionTok{before}\NormalTok{() }\KeywordTok{throws} \BuiltInTok{InterruptedException}\NormalTok{ \{}
\NormalTok{            readTry.}\FunctionTok{acquire}\NormalTok{();}
\NormalTok{            rmutex.}\FunctionTok{acquire}\NormalTok{();}
\NormalTok{            readCount++;}
            \KeywordTok{if}\NormalTok{(readCount == }\DecValTok{1}\NormalTok{) \{}
\NormalTok{                resource.}\FunctionTok{acquire}\NormalTok{();}
\NormalTok{            \}}
\NormalTok{            rmutex.}\FunctionTok{release}\NormalTok{();}
\NormalTok{            readTry.}\FunctionTok{release}\NormalTok{();}
\NormalTok{        \}}

        \AttributeTok{@Override}
        \DataTypeTok{void} \FunctionTok{after}\NormalTok{() }\KeywordTok{throws} \BuiltInTok{InterruptedException}\NormalTok{ \{}
\NormalTok{            rmutex.}\FunctionTok{acquire}\NormalTok{();}
\NormalTok{            readCount{-}{-};}
            \KeywordTok{if}\NormalTok{(readCount == }\DecValTok{0}\NormalTok{)\{}
\NormalTok{                resource.}\FunctionTok{release}\NormalTok{();}
\NormalTok{            \}}
\NormalTok{            rmutex.}\FunctionTok{release}\NormalTok{();}
\NormalTok{        \}}
\NormalTok{    \}}

    \KeywordTok{class}\NormalTok{ WriterSem }\KeywordTok{extends} \BuiltInTok{Writer}\NormalTok{ \{}

        \AttributeTok{@Override}
        \DataTypeTok{void} \FunctionTok{before}\NormalTok{() }\KeywordTok{throws} \BuiltInTok{InterruptedException}\NormalTok{ \{}
\NormalTok{            wmutex.}\FunctionTok{acquire}\NormalTok{();}
\NormalTok{            writeCount++;}
            \KeywordTok{if}\NormalTok{(writeCount == }\DecValTok{1}\NormalTok{) \{}
\NormalTok{                readTry.}\FunctionTok{acquire}\NormalTok{();}
\NormalTok{            \}}
\NormalTok{            wmutex.}\FunctionTok{release}\NormalTok{();}
\NormalTok{            resource.}\FunctionTok{acquire}\NormalTok{();}
\NormalTok{        \}}

        \AttributeTok{@Override}
        \DataTypeTok{void} \FunctionTok{after}\NormalTok{() }\KeywordTok{throws} \BuiltInTok{InterruptedException}\NormalTok{ \{}
\NormalTok{            resource.}\FunctionTok{release}\NormalTok{();}
\NormalTok{            wmutex.}\FunctionTok{acquire}\NormalTok{();}
\NormalTok{            writeCount{-}{-};}
            \KeywordTok{if}\NormalTok{(writeCount == }\DecValTok{0}\NormalTok{) \{}
\NormalTok{                readTry.}\FunctionTok{release}\NormalTok{();}
\NormalTok{            \}}
\NormalTok{            wmutex.}\FunctionTok{release}\NormalTok{();}
\NormalTok{        \}}
\NormalTok{    \}}
\NormalTok{\}}
\end{Highlighting}
\end{Shaded}

\hypertarget{rwsq---readerswriterssemqueue}{%
\subsubsection{3. RWSQ -
ReadersWritersSemQueue}\label{rwsq---readerswriterssemqueue}}

\begin{Shaded}
\begin{Highlighting}[]
\KeywordTok{class}\NormalTok{ ReadersWritersSemQueue }\KeywordTok{extends}\NormalTok{ AbstractReadersWriters}
\NormalTok{\{}
    \KeywordTok{private} \DataTypeTok{int}\NormalTok{ readCount = }\DecValTok{0}\NormalTok{;}
    \KeywordTok{private} \BuiltInTok{Semaphore}\NormalTok{ resourceAccess = }\KeywordTok{new} \BuiltInTok{Semaphore}\NormalTok{(}\DecValTok{1}\NormalTok{);}
    \KeywordTok{private} \BuiltInTok{Semaphore}\NormalTok{ readCountAccess = }\KeywordTok{new} \BuiltInTok{Semaphore}\NormalTok{(}\DecValTok{1}\NormalTok{);}
    \KeywordTok{private} \BuiltInTok{Semaphore}\NormalTok{ serviceQueue = }\KeywordTok{new} \BuiltInTok{Semaphore}\NormalTok{(}\DecValTok{1}\NormalTok{);}

    \KeywordTok{public} \FunctionTok{ReadersWritersSemQueue}\NormalTok{(}\DataTypeTok{int}\NormalTok{ repeat, }\DataTypeTok{int}\NormalTok{ sleep) \{}
        \KeywordTok{super}\NormalTok{(repeat, sleep);}
\NormalTok{    \}}

    \BuiltInTok{Reader} \FunctionTok{createReader}\NormalTok{()\{}
        \KeywordTok{return} \KeywordTok{new} \FunctionTok{ReaderSem}\NormalTok{();}
\NormalTok{    \}}
    \BuiltInTok{Writer} \FunctionTok{createWriter}\NormalTok{()\{}
        \KeywordTok{return} \KeywordTok{new} \FunctionTok{WriterSem}\NormalTok{();}
\NormalTok{    \}}

    \KeywordTok{class}\NormalTok{ ReaderSem }\KeywordTok{extends} \BuiltInTok{Reader}\NormalTok{ \{}

        \AttributeTok{@Override}
        \DataTypeTok{void} \FunctionTok{before}\NormalTok{() }\KeywordTok{throws} \BuiltInTok{InterruptedException}\NormalTok{ \{}
\NormalTok{            serviceQueue.}\FunctionTok{acquire}\NormalTok{();           }\CommentTok{// wait in line to be serviced}
\NormalTok{            readCountAccess.}\FunctionTok{acquire}\NormalTok{();        }\CommentTok{// request exclusive access to readCount}
            \CommentTok{// \textless{}ENTER\textgreater{}}
            \KeywordTok{if}\NormalTok{ (readCount == }\DecValTok{0}\NormalTok{)         }\CommentTok{// if there are no readers already reading:}
\NormalTok{                resourceAccess.}\FunctionTok{acquire}\NormalTok{();     }\CommentTok{// request resource access for readers (writers blocked)}
\NormalTok{            readCount++;                }\CommentTok{// update count of active readers}
            \CommentTok{// \textless{}/ENTER\textgreater{}}
\NormalTok{            serviceQueue.}\FunctionTok{release}\NormalTok{();           }\CommentTok{// let next in line be serviced}
\NormalTok{            readCountAccess.}\FunctionTok{release}\NormalTok{();        }\CommentTok{// release access to readCount}
\NormalTok{        \}}

        \AttributeTok{@Override}
        \DataTypeTok{void} \FunctionTok{after}\NormalTok{() }\KeywordTok{throws} \BuiltInTok{InterruptedException}\NormalTok{ \{}
\NormalTok{            readCountAccess.}\FunctionTok{acquire}\NormalTok{();        }\CommentTok{// request exclusive access to readCount}
            \CommentTok{// \textless{}EXIT\textgreater{}}
\NormalTok{            readCount{-}{-};                }\CommentTok{// update count of active readers}
            \KeywordTok{if}\NormalTok{ (readCount == }\DecValTok{0}\NormalTok{)         }\CommentTok{// if there are no readers left:}
\NormalTok{                resourceAccess.}\FunctionTok{release}\NormalTok{();     }\CommentTok{// release resource access for all}
            \CommentTok{// \textless{}/EXIT\textgreater{}}
\NormalTok{            readCountAccess.}\FunctionTok{release}\NormalTok{();        }\CommentTok{// release access to readCount}
\NormalTok{        \}}
\NormalTok{    \}}

    \KeywordTok{class}\NormalTok{ WriterSem }\KeywordTok{extends} \BuiltInTok{Writer}\NormalTok{ \{}

        \AttributeTok{@Override}
        \DataTypeTok{void} \FunctionTok{before}\NormalTok{() }\KeywordTok{throws} \BuiltInTok{InterruptedException}\NormalTok{ \{}
\NormalTok{            serviceQueue.}\FunctionTok{acquire}\NormalTok{();           }\CommentTok{// wait in line to be serviced}
            \CommentTok{// \textless{}ENTER\textgreater{}}
\NormalTok{            resourceAccess.}\FunctionTok{acquire}\NormalTok{();         }\CommentTok{// request exclusive access to resource}
            \CommentTok{// \textless{}/ENTER\textgreater{}}
\NormalTok{            serviceQueue.}\FunctionTok{release}\NormalTok{();           }\CommentTok{// let next in line be serviced}
\NormalTok{        \}}

        \AttributeTok{@Override}
        \DataTypeTok{void} \FunctionTok{after}\NormalTok{() }\KeywordTok{throws} \BuiltInTok{InterruptedException}\NormalTok{ \{}
\NormalTok{            resourceAccess.}\FunctionTok{release}\NormalTok{();}
\NormalTok{        \}}
\NormalTok{    \}}
\NormalTok{\}}
\end{Highlighting}
\end{Shaded}

\hypertarget{rwc---readerswriterscond}{%
\subsubsection{4. RWC -
ReadersWritersCond}\label{rwc---readerswriterscond}}

\begin{Shaded}
\begin{Highlighting}[]
\KeywordTok{class}\NormalTok{ ReadersWritersCond }\KeywordTok{extends}\NormalTok{ AbstractReadersWriters}
\NormalTok{\{}
    \KeywordTok{private} \DataTypeTok{final} \BuiltInTok{Lock}\NormalTok{ m = }\KeywordTok{new} \BuiltInTok{ReentrantLock}\NormalTok{();}
    \KeywordTok{private} \DataTypeTok{final} \BuiltInTok{Condition}\NormalTok{ turn = m.}\FunctionTok{newCondition}\NormalTok{();}
    \KeywordTok{private} \DataTypeTok{int}\NormalTok{ writers = }\DecValTok{0}\NormalTok{;}
    \KeywordTok{private} \DataTypeTok{int}\NormalTok{ writing = }\DecValTok{0}\NormalTok{;}
    \KeywordTok{private} \DataTypeTok{int}\NormalTok{ reading = }\DecValTok{0}\NormalTok{;}

    \KeywordTok{public} \FunctionTok{ReadersWritersCond}\NormalTok{(}\DataTypeTok{int}\NormalTok{ repeat, }\DataTypeTok{int}\NormalTok{ sleep) \{}
        \KeywordTok{super}\NormalTok{(repeat, sleep);}
\NormalTok{    \}}

    \BuiltInTok{Reader} \FunctionTok{createReader}\NormalTok{()\{}
        \KeywordTok{return} \KeywordTok{new} \FunctionTok{ReaderCond}\NormalTok{();}
\NormalTok{    \}}

    \BuiltInTok{Writer} \FunctionTok{createWriter}\NormalTok{()\{}
        \KeywordTok{return} \KeywordTok{new} \FunctionTok{WriterCond}\NormalTok{();}
\NormalTok{    \}}

    \KeywordTok{class}\NormalTok{ ReaderCond }\KeywordTok{extends} \BuiltInTok{Reader}\NormalTok{ \{}

        \AttributeTok{@Override}
        \DataTypeTok{void} \FunctionTok{before}\NormalTok{() }\KeywordTok{throws} \BuiltInTok{InterruptedException}\NormalTok{ \{}
\NormalTok{            m.}\FunctionTok{lock}\NormalTok{();}
            \KeywordTok{while}\NormalTok{ (}\DecValTok{0}\NormalTok{ \textless{} writers) \{}
\NormalTok{                turn.}\FunctionTok{await}\NormalTok{();}
\NormalTok{            \}}
\NormalTok{            reading++;}
\NormalTok{            m.}\FunctionTok{unlock}\NormalTok{();}
\NormalTok{        \}}

        \AttributeTok{@Override}
        \DataTypeTok{void} \FunctionTok{after}\NormalTok{() }\KeywordTok{throws} \BuiltInTok{InterruptedException}\NormalTok{ \{}
\NormalTok{            m.}\FunctionTok{lock}\NormalTok{();}
\NormalTok{            reading{-}{-};}
\NormalTok{            turn.}\FunctionTok{signalAll}\NormalTok{();}
\NormalTok{            m.}\FunctionTok{unlock}\NormalTok{();}
\NormalTok{        \}}
\NormalTok{    \}}

    \KeywordTok{class}\NormalTok{ WriterCond }\KeywordTok{extends} \BuiltInTok{Writer}\NormalTok{ \{}

        \AttributeTok{@Override}
        \DataTypeTok{void} \FunctionTok{before}\NormalTok{() }\KeywordTok{throws} \BuiltInTok{InterruptedException}\NormalTok{ \{}
\NormalTok{            m.}\FunctionTok{lock}\NormalTok{();}
\NormalTok{            writers++;}
            \KeywordTok{while}\NormalTok{ (}\DecValTok{0}\NormalTok{ \textless{} reading || }\DecValTok{0}\NormalTok{ \textless{} writing) \{}
\NormalTok{                turn.}\FunctionTok{await}\NormalTok{();}
\NormalTok{            \}}
\NormalTok{            writing++;}
\NormalTok{            m.}\FunctionTok{unlock}\NormalTok{();}
\NormalTok{        \}}

        \AttributeTok{@Override}
        \DataTypeTok{void} \FunctionTok{after}\NormalTok{() }\KeywordTok{throws} \BuiltInTok{InterruptedException}\NormalTok{ \{}
\NormalTok{            m.}\FunctionTok{lock}\NormalTok{();}
\NormalTok{            writing{-}{-};}
\NormalTok{            writers{-}{-};}
\NormalTok{            turn.}\FunctionTok{signalAll}\NormalTok{();}
\NormalTok{            m.}\FunctionTok{unlock}\NormalTok{();}
\NormalTok{        \}}
\NormalTok{    \}}
\NormalTok{\}}
\end{Highlighting}
\end{Shaded}

\hypertarget{wyniki}{%
\subsection{Wyniki}\label{wyniki}}

Wykresy czasu od ilości wątków czytelnika dla określonych ilości
pisarzy:

\begin{Shaded}
\begin{Highlighting}[]
\NormalTok{data \textless{}{-}}\StringTok{ }\KeywordTok{read.table}\NormalTok{(}\StringTok{"1\_writers.txt"}\NormalTok{, }\DataTypeTok{header=}\NormalTok{T)}
\KeywordTok{plot}\NormalTok{(data}\OperatorTok{$}\NormalTok{nReaders, data}\OperatorTok{$}\NormalTok{RWSRP, }\DataTypeTok{type=}\StringTok{"l"}\NormalTok{, }\DataTypeTok{main=}\StringTok{"1 writer"}\NormalTok{, }
    \DataTypeTok{ylim=}\KeywordTok{c}\NormalTok{(}\DecValTok{200}\NormalTok{, }\DecValTok{700}\NormalTok{), }\DataTypeTok{xaxs=}\StringTok{"i"}\NormalTok{, }\DataTypeTok{yaxs=}\StringTok{"i"}\NormalTok{, }\DataTypeTok{col=}\StringTok{"orange"}\NormalTok{, }
    \DataTypeTok{lwd=}\DecValTok{2}\NormalTok{, }\DataTypeTok{xlab=}\StringTok{"Readers"}\NormalTok{, }\DataTypeTok{ylab=}\StringTok{"Time [ms]"}\NormalTok{)}
\KeywordTok{lines}\NormalTok{(data}\OperatorTok{$}\NormalTok{nReaders, data}\OperatorTok{$}\NormalTok{RWSWP, }\DataTypeTok{lwd=}\DecValTok{2}\NormalTok{, }\DataTypeTok{col=}\StringTok{"red"}\NormalTok{)}
\KeywordTok{lines}\NormalTok{(data}\OperatorTok{$}\NormalTok{nReaders, data}\OperatorTok{$}\NormalTok{RWSQ, }\DataTypeTok{lwd=}\DecValTok{2}\NormalTok{, }\DataTypeTok{col=}\StringTok{"green"}\NormalTok{)}
\KeywordTok{lines}\NormalTok{(data}\OperatorTok{$}\NormalTok{nReaders, data}\OperatorTok{$}\NormalTok{RWC, }\DataTypeTok{lwd=}\DecValTok{2}\NormalTok{, }\DataTypeTok{col=}\StringTok{"blue"}\NormalTok{)}
\KeywordTok{legend}\NormalTok{(}\StringTok{"topleft"}\NormalTok{, }\DataTypeTok{legend=}\KeywordTok{c}\NormalTok{(}\StringTok{"Readers{-}P"}\NormalTok{,}\StringTok{"Writers{-}P"}\NormalTok{, }\StringTok{"FIFO"}\NormalTok{, }\StringTok{"Cond"}\NormalTok{), }
    \DataTypeTok{lwd=}\KeywordTok{c}\NormalTok{(}\DecValTok{2}\NormalTok{,}\DecValTok{2}\NormalTok{,}\DecValTok{2}\NormalTok{,}\DecValTok{2}\NormalTok{), }\DataTypeTok{col=}\KeywordTok{c}\NormalTok{(}\StringTok{"orange"}\NormalTok{,}\StringTok{"red"}\NormalTok{,}\StringTok{"green"}\NormalTok{,}\StringTok{"blue"}\NormalTok{))}
\end{Highlighting}
\end{Shaded}

\includegraphics{sprawozdanie_files/figure-latex/unnamed-chunk-1-1.pdf}

\begin{Shaded}
\begin{Highlighting}[]
\NormalTok{data \textless{}{-}}\StringTok{ }\KeywordTok{read.table}\NormalTok{(}\StringTok{"2\_writers.txt"}\NormalTok{, }\DataTypeTok{header=}\NormalTok{T)}
\KeywordTok{plot}\NormalTok{(data}\OperatorTok{$}\NormalTok{nReaders, data}\OperatorTok{$}\NormalTok{RWSRP, }\DataTypeTok{type=}\StringTok{"l"}\NormalTok{, }\DataTypeTok{main=}\StringTok{"2 writers"}\NormalTok{, }
    \DataTypeTok{ylim=}\KeywordTok{c}\NormalTok{(}\DecValTok{200}\NormalTok{, }\DecValTok{1000}\NormalTok{), }\DataTypeTok{xaxs=}\StringTok{"i"}\NormalTok{, }\DataTypeTok{yaxs=}\StringTok{"i"}\NormalTok{, }\DataTypeTok{col=}\StringTok{"orange"}\NormalTok{, }
    \DataTypeTok{lwd=}\DecValTok{2}\NormalTok{, }\DataTypeTok{xlab=}\StringTok{"Readers"}\NormalTok{, }\DataTypeTok{ylab=}\StringTok{"Time [ms]"}\NormalTok{)}
\KeywordTok{lines}\NormalTok{(data}\OperatorTok{$}\NormalTok{nReaders, data}\OperatorTok{$}\NormalTok{RWSWP, }\DataTypeTok{lwd=}\DecValTok{2}\NormalTok{, }\DataTypeTok{col=}\StringTok{"red"}\NormalTok{)}
\KeywordTok{lines}\NormalTok{(data}\OperatorTok{$}\NormalTok{nReaders, data}\OperatorTok{$}\NormalTok{RWSQ, }\DataTypeTok{lwd=}\DecValTok{2}\NormalTok{, }\DataTypeTok{col=}\StringTok{"green"}\NormalTok{)}
\KeywordTok{lines}\NormalTok{(data}\OperatorTok{$}\NormalTok{nReaders, data}\OperatorTok{$}\NormalTok{RWC, }\DataTypeTok{lwd=}\DecValTok{2}\NormalTok{, }\DataTypeTok{col=}\StringTok{"blue"}\NormalTok{)}
\KeywordTok{legend}\NormalTok{(}\StringTok{"topleft"}\NormalTok{, }\DataTypeTok{legend=}\KeywordTok{c}\NormalTok{(}\StringTok{"Readers{-}P"}\NormalTok{,}\StringTok{"Writers{-}P"}\NormalTok{, }\StringTok{"FIFO"}\NormalTok{, }\StringTok{"Cond"}\NormalTok{), }
    \DataTypeTok{lwd=}\KeywordTok{c}\NormalTok{(}\DecValTok{2}\NormalTok{,}\DecValTok{2}\NormalTok{,}\DecValTok{2}\NormalTok{,}\DecValTok{2}\NormalTok{), }\DataTypeTok{col=}\KeywordTok{c}\NormalTok{(}\StringTok{"orange"}\NormalTok{,}\StringTok{"red"}\NormalTok{,}\StringTok{"green"}\NormalTok{,}\StringTok{"blue"}\NormalTok{))}
\end{Highlighting}
\end{Shaded}

\includegraphics{sprawozdanie_files/figure-latex/unnamed-chunk-2-1.pdf}

\begin{Shaded}
\begin{Highlighting}[]
\NormalTok{data \textless{}{-}}\StringTok{ }\KeywordTok{read.table}\NormalTok{(}\StringTok{"5\_writers.txt"}\NormalTok{, }\DataTypeTok{header=}\NormalTok{T)}
\KeywordTok{plot}\NormalTok{(data}\OperatorTok{$}\NormalTok{nReaders, data}\OperatorTok{$}\NormalTok{RWSRP, }\DataTypeTok{type=}\StringTok{"l"}\NormalTok{, }\DataTypeTok{main=}\StringTok{"5 writers"}\NormalTok{, }
    \DataTypeTok{ylim=}\KeywordTok{c}\NormalTok{(}\DecValTok{1000}\NormalTok{, }\DecValTok{2000}\NormalTok{), }\DataTypeTok{xaxs=}\StringTok{"i"}\NormalTok{, }\DataTypeTok{yaxs=}\StringTok{"i"}\NormalTok{, }\DataTypeTok{col=}\StringTok{"orange"}\NormalTok{, }
    \DataTypeTok{lwd=}\DecValTok{2}\NormalTok{, }\DataTypeTok{xlab=}\StringTok{"Readers"}\NormalTok{, }\DataTypeTok{ylab=}\StringTok{"Time [ms]"}\NormalTok{)}
\KeywordTok{lines}\NormalTok{(data}\OperatorTok{$}\NormalTok{nReaders, data}\OperatorTok{$}\NormalTok{RWSWP, }\DataTypeTok{lwd=}\DecValTok{2}\NormalTok{, }\DataTypeTok{col=}\StringTok{"red"}\NormalTok{)}
\KeywordTok{lines}\NormalTok{(data}\OperatorTok{$}\NormalTok{nReaders, data}\OperatorTok{$}\NormalTok{RWSQ, }\DataTypeTok{lwd=}\DecValTok{2}\NormalTok{, }\DataTypeTok{col=}\StringTok{"green"}\NormalTok{)}
\KeywordTok{lines}\NormalTok{(data}\OperatorTok{$}\NormalTok{nReaders, data}\OperatorTok{$}\NormalTok{RWC, }\DataTypeTok{lwd=}\DecValTok{2}\NormalTok{, }\DataTypeTok{col=}\StringTok{"blue"}\NormalTok{)}
\KeywordTok{legend}\NormalTok{(}\StringTok{"topleft"}\NormalTok{, }\DataTypeTok{legend=}\KeywordTok{c}\NormalTok{(}\StringTok{"Readers{-}P"}\NormalTok{,}\StringTok{"Writers{-}P"}\NormalTok{, }\StringTok{"FIFO"}\NormalTok{, }\StringTok{"Cond"}\NormalTok{), }
    \DataTypeTok{lwd=}\KeywordTok{c}\NormalTok{(}\DecValTok{2}\NormalTok{,}\DecValTok{2}\NormalTok{,}\DecValTok{2}\NormalTok{,}\DecValTok{2}\NormalTok{), }\DataTypeTok{col=}\KeywordTok{c}\NormalTok{(}\StringTok{"orange"}\NormalTok{,}\StringTok{"red"}\NormalTok{,}\StringTok{"green"}\NormalTok{,}\StringTok{"blue"}\NormalTok{))}
\end{Highlighting}
\end{Shaded}

\includegraphics{sprawozdanie_files/figure-latex/unnamed-chunk-3-1.pdf}

\begin{Shaded}
\begin{Highlighting}[]
\NormalTok{data \textless{}{-}}\StringTok{ }\KeywordTok{read.table}\NormalTok{(}\StringTok{"10\_writers.txt"}\NormalTok{, }\DataTypeTok{header=}\NormalTok{T)}
\KeywordTok{plot}\NormalTok{(data}\OperatorTok{$}\NormalTok{nReaders, data}\OperatorTok{$}\NormalTok{RWSRP, }\DataTypeTok{type=}\StringTok{"l"}\NormalTok{, }\DataTypeTok{main=}\StringTok{"10 writers"}\NormalTok{, }
    \DataTypeTok{ylim=}\KeywordTok{c}\NormalTok{(}\DecValTok{2000}\NormalTok{, }\DecValTok{3000}\NormalTok{), }\DataTypeTok{xaxs=}\StringTok{"i"}\NormalTok{, }\DataTypeTok{yaxs=}\StringTok{"i"}\NormalTok{, }\DataTypeTok{col=}\StringTok{"orange"}\NormalTok{, }
    \DataTypeTok{lwd=}\DecValTok{2}\NormalTok{, }\DataTypeTok{xlab=}\StringTok{"Readers"}\NormalTok{, }\DataTypeTok{ylab=}\StringTok{"Time [ms]"}\NormalTok{)}
\KeywordTok{lines}\NormalTok{(data}\OperatorTok{$}\NormalTok{nReaders, data}\OperatorTok{$}\NormalTok{RWSWP, }\DataTypeTok{lwd=}\DecValTok{2}\NormalTok{, }\DataTypeTok{col=}\StringTok{"red"}\NormalTok{)}
\KeywordTok{lines}\NormalTok{(data}\OperatorTok{$}\NormalTok{nReaders, data}\OperatorTok{$}\NormalTok{RWSQ, }\DataTypeTok{lwd=}\DecValTok{2}\NormalTok{, }\DataTypeTok{col=}\StringTok{"green"}\NormalTok{)}
\KeywordTok{lines}\NormalTok{(data}\OperatorTok{$}\NormalTok{nReaders, data}\OperatorTok{$}\NormalTok{RWC, }\DataTypeTok{lwd=}\DecValTok{2}\NormalTok{, }\DataTypeTok{col=}\StringTok{"blue"}\NormalTok{)}
\KeywordTok{legend}\NormalTok{(}\StringTok{"topleft"}\NormalTok{, }\DataTypeTok{legend=}\KeywordTok{c}\NormalTok{(}\StringTok{"Readers{-}P"}\NormalTok{,}\StringTok{"Writers{-}P"}\NormalTok{, }\StringTok{"FIFO"}\NormalTok{, }\StringTok{"Cond"}\NormalTok{), }
    \DataTypeTok{lwd=}\KeywordTok{c}\NormalTok{(}\DecValTok{2}\NormalTok{,}\DecValTok{2}\NormalTok{,}\DecValTok{2}\NormalTok{,}\DecValTok{2}\NormalTok{), }\DataTypeTok{col=}\KeywordTok{c}\NormalTok{(}\StringTok{"orange"}\NormalTok{,}\StringTok{"red"}\NormalTok{,}\StringTok{"green"}\NormalTok{,}\StringTok{"blue"}\NormalTok{))}
\end{Highlighting}
\end{Shaded}

\includegraphics{sprawozdanie_files/figure-latex/unnamed-chunk-4-1.pdf}

\hypertarget{wnioski}{%
\subsection{Wnioski}\label{wnioski}}

Metody:

\begin{itemize}
\item
  Semaphore readers-preference
\item
  Semaphore writers-preference
\item
  Conditional variables
\end{itemize}

Zachowują się w sposób bardzo zbliżony. Przy większych ilościach
czytelników widzimy jednak, że najszybszy okazuje się
readers-preference, a najwolniejszy writers-preference.

Zdecydowanie odbiega od nich metoda FIFO z użyciem semaforów - czasy są
dużo wyższe i bardzo rozbieżne.

\end{document}
